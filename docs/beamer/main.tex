\documentclass[aspectratio=169]{beamer}
\usepackage{listings}
\usepackage{listings-rust}

%\usetheme{Warsaw}
\usetheme{CambridgeUS}
%\usetheme{Antibes}
%\usecolortheme{crane}
%\usecolortheme{seahorse}
%\usecolortheme{wolverine}
\usecolortheme{beaver}
%\usecolortheme{whale}


\title[Fault Injection Environment per Applicazione ridondata]
{\Huge Realizzazione di un ambiente di Fault Injection
per applicazione ridondata
}
\subtitle{Progetto per il corso di \textit{Programmazione di Sistema}}

\author[C.Migliaccio, F.Pretini, A.Scavone, M.Viglino]
{Carlo~Migliaccio\and 
Federico Pretini\and  
Alessandro Scavone \and 
Mattia Viglino }

\institute[PoliTO]
{Corso di Laurea Magistrale in Ingegneria Informatica \\
Politecnico di Torino}
\date[Gennaio 2025 AA 2024/2025]{Gennaio 2025\\
Anno Accademico 2024/2025}

\AtBeginSection[]
{
  \begin{frame}
    \frametitle{Fault Injection Environment per Applicazione ridondata}
    \tableofcontents[currentsection]
  \end{frame}
}

\AtBeginSubsection[]
{   \begin{frame}
        \frametitle{Fault Injection Environment per Applicazione ridondata}
        \tableofcontents[currentsubsection]
    \end{frame}
}

%\logo{\includegraphics[height=1.7cm]{polito-logo.png}}


\begin{document}

\defverbatim[colored]\hardened{
    \begin{lstlisting}[language=Rust,style=boxed]
#[derive(Clone,Copy)]
pub struct Hardened<T>{
    cp1: T, 
    cp2: T,
}\end{lstlisting}
}

\defverbatim[colored]\hardenedimpl{
    \begin{lstlisting}[language=Rust, style=boxed]
impl<T> Hardened<T>{
    ... //Implementation
}
    \end{lstlisting}
}

\defverbatim[colored]\traitone{
    \begin{lstlisting}[language=Rust,style=boxed]
impl<T> Sub for Hardened<T>
where T:Sub<Output=T>+PartialEq+Eq+Debug+Copy+Clone{
    type Output=Result<Hardened<T>,IncoherenceError>;
    fn sub(self, rhs: Self) -> Self::Output {
        if self.incoherent() || rhs.incoherent(){
            return Err(IncoherenceError::Generic)
        }
        Ok(Self{
            cp1: self.cp1 - rhs.cp1,
            cp2: self.cp2 - rhs.cp2,
        })
    }
}       
    \end{lstlisting}
}

\defverbatim[colored]\incerror{
    \begin{lstlisting}[language=Rust, style=boxed]
#[derive(Error, Debug, Clone)]
pub enum IncoherenceError{
    #[error("IncoherenceError::AssignFail: assignment failed")]
    AssignFail,
    #[error("IncoherenceError::AddFail: due to incoherence add failed")]
    AddFail,
    #[error("IncoherenceError::MulFail: due to incoherence mul failed")]
    MulFail,
    #[error("IncoherenceError::Generic: generic incoherence error")]
    Generic
}       
    \end{lstlisting}
}


\frame{\setbeamertemplate{logo}{}\titlepage}

\begin{frame}
    \frametitle{Fault Injection Environment per Applicazione ridondata}
    \tableofcontents
    \end{frame}

    \section{Introduzione}
    

    \begin{frame}
        \frametitle{Prova uso \texttt{listings}}
        \framesubtitle{...di Alex Scavonx}
        Il tipo \alert{\texttt{Hardened}} è definito come segue: 
        \hardened
        Inoltre è implementato come segue:
        \hardenedimpl
    \end{frame}

    \begin{frame}
        \frametitle{Implementazione tratto}
        \traitone
    \end{frame}

    \begin{frame}
        \frametitle{Tipo \texttt{IncoherenceError}}
        \incerror
    \end{frame}

        \begin{frame}
            \frametitle{Introduzione all'argomento}
            \begin{block}{Blocco}
                Esempio di utilizzo di blocchi e elenchi numerati
            \end{block}      
            \begin{enumerate}
                \item Primo item
                \item Secondo item
            \end{enumerate}
        \end{frame}

ì    \begin{frame}
        \frametitle{Tre regole}
        \framesubtitle{Sottotitolo}
        \begin{alertblock}{Data redundancy}
            Tre \textbf{regole fondamentali} per l'irrobustimento del codice: 
            \begin{enumerate}
                \item Ogni operazione di lettura deve essere preceduta da un controllo di consistenza;
                \item Ogni scrittura deve essere eseguita  sulle due copie
            \end{enumerate}
        \end{alertblock}
    \end{frame}

    

    \section{Irrobustimento codice}

    
    \begin{frame}
    \frametitle{Sample frame title}
    This is some text in the first frame. This is some text in the first frame. This is some text in the first frame.\\ \pause
    This is an example of an \alert{highlighted} text
    \end{frame}

    \begin{frame}
        \frametitle{Trasformata di Laplace}
        La \textbf{Trasformata di Laplace} è una trasformata integrale dal dominio $t\in\mathbb{R}$ al dominio $s\in\mathbb{C}$. Riportiamo per semplicità di seguito la sua definizione.

        \begin{equation}
            \mathcal{L}\{f(t)\}(s) = \int_{0}^{+\infty} {f(t) e^{-st}} dt 
        \end{equation}

        \begin{definition}[\textbf{State  Space Representation}]
            Un sistema lineare tempo invariante (LTI) può avere nello spazio di stato la seguente rappresentazione. 
            \begin{equation}
                \begin{aligned}
                    &x(t)=Ax(t)+Bu(t)\\
                    &y(t)=Cx(t)+Du(t)
                \end{aligned}
            \end{equation}
        \end{definition}

        \begin{example}
            Questo è un esempio
        \end{example}
    \end{frame}

    \begin{frame}
        \begin{alertblock}{Importante!}
            Questo è un alert block
        \end{alertblock}
    \end{frame}

    \subsection{Creazione del tipo \texttt{Hardened<T>}}
    \begin{frame}
        \frametitle{Regression form}
        \begin{definition}[Feasible Parameter Set]
            The \textbf{The Feasible Parameter Set (FPS)} is the set of parameters $\theta$ which are consistent with the a-priori and a-posteriori information.
            \begin{equation}
                \begin{aligned}
                    \mathcal{D}_\theta=\{
                        \theta\in\mathbb{R}^p: 
                        \tilde{y}(k)=&-\theta_1{y(k-1)}+
                        -\theta_2{y(k-2)}+\theta_3{u(k)}\\
                        &+\theta_4{u(k-1)}+\theta_5{u(k-2)}+e(k), \ k=3,...,H \\
                        & \vert e(k) \vert \le \Delta_e, \
                        k=1,...,H
                     \}
                \end{aligned}
            \end{equation}
        \end{definition}
        Under the assumption of \textbf{Equation Error (EE)} noise structure the computation of the PUIs becomes a couple of LP problems. 
        \begin{align}
            &PUI_{\theta_j}=[\underline{\theta}_j,\overline{\theta}_j],\\
            &\underline{\theta}_j=\min_{\theta\in\mathcal{D}_\theta} {\theta_j}, \quad
            \overline{\theta}_j=\max_{\theta\in\mathcal{D}_\theta} {\theta_j}
        \end{align}
        
    \end{frame}
    \subsection{Refactoring degli algoritmi di base}

    
    \section{Fault Injection Environment}

    \subsection{Fault List Manager}
    \subsection{Injector}
    \subsection{Analizer}

    \begin{frame}
        \frametitle{Sample frame title}
        This is some text in the first frame. This is some text in the first frame. This is some text in the first frame.
        \end{frame}

\end{document}