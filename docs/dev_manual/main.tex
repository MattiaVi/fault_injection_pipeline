\documentclass{article}
\usepackage{lipsum}
\usepackage{authblk}
\usepackage{algorithm}
\usepackage{algpseudocode}
\usepackage{dirtree}
\usepackage{pifont}
\usepackage{enumitem}
\usepackage[utf8]{inputenc}
\usepackage{subcaption}
\usepackage{caption}
\usepackage{amssymb}
\usepackage{amsmath}
\usepackage[hidelinks]{hyperref}
\usepackage[svgnames]{xcolor}
\usepackage[a4paper,total={7in, 8in}]{geometry}
\usepackage{bm}
\usepackage{centernot}
\usepackage{titling}
\usepackage{color}
\usepackage{listings}
\usepackage{listings-rust}
\usepackage{graphicx} % Required for inserting images
\usepackage{tikz}
\tikzstyle{mybox} = [draw=black, thin, rectangle, rounded corners, inner ysep=5pt, inner xsep=5pt, fill=orange!20]


%Per disegnare il box
%\hspace*{-5mm}
%\begin{tikzpicture}
%\node [mybox] (box){%
%    \begin{minipage}{.96\textwidth}     %Larghezza del box
            %Qui testo
%    \end{minipage}
%};
%\end{tikzpicture}%



\title{\textbf{ Realizzazione di un ambiente di Fault Injection
per applicazione ridondata
[Manuale di Sviluppo]}}
\author{Carlo Migliaccio}
\author{Federico Pretini}
\author{Alessandro Scavone}
\author{Mattia Viglino}
\affil[1]{\small{Laurea Magistrale in Ingegneria Informatica, Politecnico di Torino}}

\date{Febbraio 2025}

\pagestyle{headings}

\begin{document}

\counterwithin{figure}{section}
\counterwithin{equation}{section}
\renewcommand{\labelenumii}{\arabic{enumi}.\arabic{enumii}}

\maketitle
\thispagestyle{empty}
\vspace{-0.8cm}
\begin{abstract}
    \noindent
    La \textit{dependability} e la \textit{tolleranza ai guasti} sono temi di fondamentale importanza nello sviluppo di applicazioni e sistemi safety critical. Le metriche classiche adottate per quantificare affidabilità e robustezza (eg. MTBF) richiedono l'osservazione del sistema per periodi lunghi. Talvolta si vuole accelerare (artificialmente)  l'occorrenza dei guasti tramite tecniche (hardware o software) di fault injection. Il lavoro proposto ha un duplice obiettivo: (i) l'implementazione di un'applicazione \textit{fault tolerant} tramite modifiche sistematiche del codice sorgente in linguaggio \texttt{Rust} (ispirate a \cite{rebaudengo1999soft}); (ii) la realizzazione di un ambiente di fault injection per testare l'efficacia delle trasformazioni effettuate. Il presente \textsc{Manuale di Sviluppo} fornisce una descrizione dei passaggi fondamentali dell'implementazione, integrata -- dove necessario -- con brevi riferimenti teorici.
\end{abstract}
\tableofcontents

\newpage

\section{Introduzione}
Nella società attuale si fa, in generale, un utilizzo capillare di sistemi computerizzati, questi nello specifico sono coinvolti in settori in cui un guasto al sistema potrebbe essere critico, mettendo potenzialmente a rischio vite umane. L'implementazione di sistemi \textit{safety-critical} espone lo sviluppatore ad affrontare problemi non trascurabili che coinvolgono la valutazione della \textbf{dependability} \cite{noauthor_dependability_2024} e della \textbf{tolleranza ai guasti} (\textit{fault tolerance}). Le tecniche di test standard e l'utilizzo di benchmark non bastano in quest'ambito, in quanto per valutare certi aspetti del sistema (quali la dependability) bisognerebbe osservarne il comportamento dopo che il guasto si verifica. \\
In ambito 'fault-tolerance' vengono utilizzate delle metriche specifiche per quantificare affidabilità e robustezza del sistema in analisi. Per citarne una significativa, riportiamo l'\textit{MTBF} (Mean Time Between Failure); questa per i sistemi concepiti, per essere tolleranti ai guasti, potrebbe essere associata ad un periodo di tempo molto lungo, anche anni! Da tempo la ricerca va nella direzione di trovare un modo per 'accelerare' in simulazione l'occorrenza di questi guasti/difetti prima che accadano naturalmente, dal momento che questa lunga latenza rende difficile anche solo identificare la causa di un potenziale difetto/guasto. In molti casi l'approccio utilizzato è quello basato su esperimenti di \textbf{fault injection}, che -- come riportato in \cite{depend} -- vengono usati non solo durante l'implementazione, ma anche durante la fase prototipale e operativa, permettendo quindi di coprire un ampio spettro di casistiche.\\
La survey \cite{hsueh1997fault} individua principalmente due grandi famiglie di \textit{tecniche di fault injection}:
(i) \textsc{Fault Injection Hardware} di cui non ci occupiamo; (ii) \textsc{Fault Injection Software}, è la famgilia  di tecniche che negli ultimi anni ha attirato l'interesse dei ricercatori in quanto tali metodi non richiedono hardware costoso. Inoltre nei contesti in cui il target sia un \textit{applicativo} o, ancora peggio, il \textit{sistema operativo}, costituiscono l'unica scelta.

\noindent
Nel lavoro qui presentato si adotta un approccio \textit{software} che si pone principalmente \textbf{due obiettivi}: 
\begin{enumerate}
    \itemsep-0.3em
    \item La modifica del codice per \textit{casi di studio scelti} volta ad introdurre ridondanza nel sistema tramite la \textbf{duplicazione di tutte le variabili}; 
    \item La realizzazione di un \textbf{ambiente software di fault injection} per simulare l'occorrenza di guasti nel sistema irrobustito e valutarne l'\textit{affidabilità}. Il \textbf{modello di fault} analizzato è il \textit{transient single bit-flip fault}, su cui si basano molti tool di fault injection. 
\end{enumerate}

\noindent 
Il presente documento si pone l'obiettivo di evidenziare i passaggi salienti dell'implementazione delle tecniche di \textit{Software fault tolerance} e dell'\textit{ambiente di fault injection} riportando schemi e \textit{code snippet}  che insieme al codice sorgente -- in \textbf{linguaggio Rust} -- permettono di seguire le diverse fasi del lavoro svolto.\\
Nella \textbf{Sezioni \ref{sec:hardened}-\ref{sec:transf_impl}} viene introdotto e successivamente implementato un \textbf{set di trasformazioni del codice} sfruttando parte dei concetti presentati in \cite{rebaudengo1999soft}. Si anticipa che l'implementazione di queste regole è integrata nella realizzazione di un nuovo tipo generico(\texttt{Hardened<T>}) di cui si descrivono gli aspetti chiave. La \textbf{Sezione \ref{sec:CaseStudies}} fornisce un'analisi comparativa del codice non irrobustito rispetto a quello che utilizza le variabili di tipo generico \texttt{Hardened<T>}. Nella \textbf{Sezione \ref{sec:Fault_Env}} viene definita la struttura dell'ambiente di fault injection, mentre le \textbf{Sezioni \ref{sec:FLM}, \ref{sec: Injector}, \ref{sec:analizer}} si occupano di analizzarne i componenti. Infine la \textbf{Sezione \ref{sec:Struttura_Codice}} descrive brevemente la struttura del codice sorgente e dei test d'unità.
        
\part{Irrobustimento del codice}
\section{Tipo \texttt{Hardened<T>} e irrobustimento del codice} \label{sec:hardened}

\subsection{Tre regole per la trasformazione del codice}

\begin{lstlisting}[language=Rust, style=boxed]
struct Hardened<T>{
    cp1: T, 
    cp2: T
}
\end{lstlisting}


\begin{lstlisting}[language=Rust, style=boxed]
impl<T> Hardened<T>{
    
}
\end{lstlisting}     
\section{Casi di studio}\label{sec:CaseStudies}
I tre casi di studio presi in considerazione sono: 
\begin{itemize}
    \item Selection Sort
    \item Bubble Sort
    \item Moltiplicazioni tra matrici
\end{itemize}
Per ciascuno di essi vengono implementate due versioni:
\begin{itemize}
    \item La versione Non Hardened, implementazione standard del caso studio.
    \item La versione Hardened, basata sul tipo Hardeded e sulle tre regole di trasformazione del codice precedentemente enunciate.
\end{itemize}
Essendo la ridondanza dei dati e le operazioni associate integrate nell'implementazione del tipo Hardened, per tutti i casi studio le due versioni non differiscono nella strategia algoritmica, ma solamente in piccole varaiazioni sintattiche.
\\ La logica di ordinamento e di calcolo infatti rimane sempre la stessa. \\ \\
Di seguito vengono riportati i codici dei casi studio.
\subsection{Selection Sort}
L'algoritmo di ordinamento selection sort ordina un vettore trovando il minimo in ogni iterazione e scambiandolo con l'elemento corrente.
\\ \\Versione Non Hardened:
\begin{lstlisting}[language=rust, style=boxed] 
    pub fn selection_sort(mut vet:Vec<i32>)->Vec<i32>{

    let n:usize = vet.len();
    let mut j=0;
    let mut min=0;

    let mut i=0;
    while i< n -1{
        min=i;
        j=i+1;

        while j< n {
            if vet[j] < vet[min]{ min=j; }
            j = j+1;
        }
        vet.swap(min,i);
        i=i+1;
    }
    vet
}
\end{lstlisting}
Versione Hardened:
\begin{lstlisting}[language=rust, style=boxed] 
pub fn selection_sort(vet: &mut Vec<Hardened<i32>>)->Result<(), IncoherenceError>{

    let n:Hardened<usize> = vet.len().into();
    let mut j= Hardened::from(0);
    let mut min = Hardened::from(0);

    let mut i= Hardened::from(0);
    while i<(n -1)?{
        min.assign(i)?;                 
        j.assign((i+1)?)?;        
      
        while j< n {
            if vet[j]<vet[min]  {   min.assign(j)?; }
            j.assign((j+1)?)?;
        }
        vet.swap(i.inner()?, min.inner()?);
        i.assign((i+1)?)?;
    }
    Ok(())
}
\end{lstlisting}

\subsection{Bubble Sort}
L'algoritmo di ordinamento bubble sort è un algortimo che confronta e scambia elementi adiacenti finché i dati non risultano ordinati. La variabile swapped ottimizza il processo in quanto se non vengono effettuati scambi in un ciclo, significa che il vettore è già ordinato.\\
\\Versione Non Hardened:
\begin{lstlisting}[language=rust, style=boxed] 
    pub fn bubble_sort(mut vet: Vec<i32>) -> Vec<i32> {
        let n:usize = vet.len();
        let mut i = 0;
    
        while i < n {
            let mut swapped = false;
            let mut j = 0;
    
            while j < n - i - 1 {
                if vet[j] > vet[j + 1] {
                    vet.swap(j, j + 1);
                    swapped = true;
                }
                j += 1;
            }
            if !swapped {
                break;
            }
            i += 1;
        }
        vet
    }    
\end{lstlisting}
Versione Hardened:
\begin{lstlisting}[language=rust, style=boxed] 
pub fn bubble_sort(vet: &mut Vec<Hardened<i32>>) -> Result<(), IncoherenceError> {

    let n = Hardened::from(vet.len());
    let mut i = Hardened::from(0);

    while i < n {
        let mut swapped = Hardened::from(false);
        let mut j = Hardened::from(0);

        while j < ((n - i)? - 1)? {
            if vet[j].inner()? > vet[(j + 1)?].inner()? {
                vet.swap(j.inner()?, (j + 1)?.inner()?);
                swapped = Hardened::from(true);
            }
            j.assign((j + 1)?)?;
        }
        if !swapped.inner()? {
            break;
        }
        i.assign((i + 1)?)?;
    }
    Ok(())
}
\end{lstlisting}



\subsection{Moltiplicazioni tra matrici}
L'algoritmo moltiplica due matrici quadrate, calcolando ogni elemento come il prodotto scalare delle righe di una matrice e delle colonne dell'altra.
\\ \\Versione Non Hardened:
\begin{lstlisting}[language=rust, style=boxed] 
    pub fn matrix_multiplication(a: Vec<Vec<i32>>, b: Vec<Vec<i32>>) -> Vec<Vec<i32>> {
        let size:usize = a.len();
        let mut result: Vec<Vec<i32>> = Vec::new();
    
        let mut i = 0;
        let mut j = 0;
        let mut k = 0;
    
        while i < size {
            let mut row: Vec<i32> = Vec::new();
            j = 0;
    
            while j < size {
                let mut acc = 0;
                k = 0;
    
                while k < size {
                    acc += a[i][k] * b[k][j];
                    k += 1;
                }
                row.push(acc); 
                j += 1;
            }
            result.push(row);
            i += 1;
        }
        result
    }    
\end{lstlisting}
Versione Hardened:
\begin{lstlisting}[language=rust, style=boxed] 
    pub fn matrix_multiplication(a: &Vec<Vec<Hardened<i32>>>, b: &Vec<Vec<Hardened<i32>>>) -> Result<Vec<Vec<Hardened<i32>>>, IncoherenceError> {
    
        let size = Hardened::from(a.len());
        let mut result: Vec<Vec<Hardened<i32>>> = Hardened::from_mat(Vec::new());
    
        let mut i = Hardened::from(0);
        let mut j = Hardened::from(0);
        let mut k = Hardened::from(0);
    
        while i < size {
            let mut row: Vec<Hardened<i32>> = Vec::new();
            j.assign(Hardened::from(0))?;
    
            while j < size {
                let mut acc = Hardened::from(0);
                k.assign(Hardened::from(0))?;
    
                while k < size {
                    acc.assign((acc + Hardened::from(a[i.inner()?][k.inner()?].inner()?   *   b[k.inner()?][j.inner()?].inner()?) )? )?;
                    k.assign((k + 1)?)?;
                }
                row.push(acc); 
                j.assign((j + 1)?)?;
            }
            result.push(row); 
            i.assign((i + 1)?)?;
        }
        Ok(result)
    }
\end{lstlisting}    
\part{Ambiente di fault injection}
\section{Implementazione e descrizione dell'ambiente}\label{sec:Fault_Env}

\begin{figure}[h]
    \centering
    \includegraphics[scale=0.5]{img/pipeline.png}
    \caption{Struttura della pipeline}
\end{figure}

La fonte principale di ispirazione per la realizzazione della parte di applicazione devota ai fault injection è stato \cite{benso_fault_1998}.
Si possono individuare in questa parte principalmente tre sezioni: 
\begin{enumerate}
    \itemsep-0.3em
    \item \textsc{Fault List Manager} (FLM): genera la \textbf{lista dei fault} \textit{fault list} da iniettare nel \textbf{programma target}; 
    \item \textsc{Injector} \textit{(Fault injection Manager)} (FIM): \textbf{inietta i fault} nel programma target; 
    \item \textsc{Result Analyzer}: raccoglie i \textbf{risultati}, ne calcola degli \textbf{aggregati} e genera un \textbf{report} riferito al singolo esperimento di fault injection.
\end{enumerate}

Al fine di parallelizzare i compiti nei vari livelli si è deciso di adottare un \textit{pattern architetturale} che in letteratura è noto come \textbf{\textsc{pipes and filters}} (si veda il libro \cite{schmidt2013pattern} per una trattazione approfondita). Questo perché, come risulta evidente dall'introduzione, esistono nel progetto \textbf{fasi separate di elaborazione}. Inoltre tale struttura si presta molto bene allo sviluppo dell'applicativo in un team costituito da più persone. Per maggiore chiarezza dei concetti esposti si riporta qui una possibile definizione del pattern tratta da \cite{schmidt2013pattern}: 
\begin{quotation}
    \textit{
        \noindent
"The Pipes and Filters architectural pattern provides a structure for
systems that process a stream of \textbf{data}. Each processing step is
encapsulated in a \textbf{filter component}. Data is passed through \textbf{pipes}
between adjacent filters. Recombining filters allows you to build
families of related systems"}
\end{quotation}
Di seguito si riporta 
la versione modificata della pipeline in cui dove vengono aggiunti anche i componenti citati nella definizione. 

\begin{figure}[h]
    \centering
    \includegraphics[scale=0.5]{img/pipeline_mapped.png}
    \caption{Pattern architetturale \textbf{pipes and filters}}
\end{figure}

\subsection{Implementazione in Rust}
Ogni componente della pipeline mostrata trova implementazione in Rust in opportuni componenti software. La \textsc{sorgente dei dati} è costituita da un sottosistema che esegue l'\textbf{analisi statica automatica} del \textbf{programma target} producendo un primo report (\textit{Static analysis result)} (data preparation) che viene utilizzato a sua volta da una routine preposta alla generazione della \textbf{generare la fault list}. I \textsc{tre filtri}, invece, sono delle subroutine allocate in appositi moduli (vedi \textbf{Sezione \ref{sec:Struttura_Codice}} per la struttura del codice sorgente). Le due \textsc{pipeline} tramite le quali comunicano i tre stage sono implementate tramite 2 canali \textit{multiple producer single consumer} \texttt{(mpsc)}. Ogni filtro\footnote{
    Questi in letteratura sono noti come \textit{filtri attivi} in quanto rispettano il seguente principio: "[...] the filter is active in a loop, pulling its input from
    and pushing its output down the pipeline."(cfr. \cite{schmidt2013pattern})
} utilizza come struttura di \textbf{programmazione concorrente} i thread.
In conclusione di questa sezione forniamo: (i) lo snippet di codice che riporta la \textbf{funzione di setup} della pipeline; una \textbf{tabella riassuntiva} (Tabella \ref{setup}) di tutti gli elementi della pipeline con annessa descrizione, ruolo e sezione di codice a cui afferisce.\newpage

\begin{lstlisting}[language=rust, style=boxed][h]
pub fn fault_injection_env(fault_list: String,   //Data Source: Fault List
                        target: String,          //Data Source: Target program
                        file_path: String,       //Data Sink: Report
                        data: Data<i32>) {       //Data Source: Data

    let (tx_chan_fm_inj, rx_chan_fm_inj) = channel();           //pipe FLM-FIM
    let (tx_chan_inj_anl, rx_chan_inj_anl) = channel();         //pipe FIM-Analyzer

    fault_manager(tx_chan_fm_inj,fault_list);
    injector_manager(rx_chan_fm_inj, tx_chan_inj_anl, target, data);
    analyzer(rx_chan_inj_anl,file_path);
}
\end{lstlisting}

\begin{table}[h] 
    \centering
    \small
    \begin{tabular}{p{4cm} p{2.5cm} p{8.5cm}}
        \toprule[1.5px]
        \textbf{\large Componente}&\textbf{\large Ruolo}&\textbf{\large Descrizione}\\
        \midrule[1.5px]
        \textsc{Static analysis result}&Data source&Dato il programma target lo analizza estrandone le informazioni sulle variabili (tipo, nome, dimensione...) ed  effettua un conteggio delle istruzioni. Il sottomodulo \texttt{mod static\_analysis} contiene tutte le funzioni collegate a questo task.\\
        \hline
        \textsc{Target program}&Data source&
            Codice sorgente dei casi di studio oggetto dell'analisi: irrobustimento dei dati e fault injection. La directory \texttt{fault\_list\_manager/file\_fault\_list} contiene i file \texttt{*.rs} input dell'analisi statica.\\
            \hline
            Data&Data source&Sono i dati su cui lavorano gli algoritmi selezionati come casi di studio. I dati possono essere prelevati dal file \texttt{data/input.txt} o dai dataset \texttt{data/dataset.txt}.\\
            \hline
        \textsc{Fault list}&Data source& Sfruttando il report prodotto dall'analisi statica viene generata la fault list contenente un numero di entry scelto dall'utente o prefissato. Il modulo \texttt{mod fault\_list\_manager} contiene le funzioni adibite a tale compito.\\
        \midrule
        \textsc{Fault list manager}&Filter&Stage della pipeline che scorre la fault list, preleva una fault entry per volta e la spedisce nella \textit{pipe} verso l'iniettore incapsulato in una struttura di tipo \texttt{FaultListEntry}. (\texttt{mod fault\_list\_manager})\\\hline
        \textsc{Injector}&Filter&Prelevando dal canale le \texttt{FaultListEntry} esegue gli esperimenti di iniezione e spedisce i risultati grezzi da analizzare a valle nella pipeline. \texttt{(mod injector)}\\\hline
        \textsc{Analizer}&Filter&Stage della pipeline che riceve i risultati dei test dall'iniettore, calcola delle statistiche e produce il report finale dell'esperimento. (\texttt{mod analizer})\\\hline
        \textsc{Results}&Data sinks&File contenente il report dell'esperimento di fault injection con grafici e tabelle. Viene memorizzato nella directory \texttt{result}.\\
        \midrule
        \textsc{Canali} \texttt{mpsc}&pipes&Costituiscono il collante degli stage della pipeline. Le istruzioni 6-7 creano le estremità delle due pipeline, queste vengono poi distribuite ai vari filtri.\\
        \bottomrule[1.5px]
   \end{tabular}
    \caption{Tabella riassuntiva \textit{Fault injection environment}}
    \label{setup}
\end{table}     
\newpage
\section{Data source} \label{sec:data_source}
In questa sezione sono descritti i vari moduli facenti parte della data source.

\subsection{Origine dei dati per gli algoritmi}
Di seguito vengono presentate le sorgenti di provenienza dei dati, utilizzati per eseguire gli algoritmi del programma. I file presentati di seguito si trovano nella cartella \texttt{src/data} \\

\noindent I dati possono essere vettori o matrici e sulla base delle scelta utente vengono estratti da:  
\begin{itemize}
    \item \textbf{Dataset dei vettori}: un dataset predefinito di vettori causuali.
    \item \textbf{Dataset delle matrici}: un dataset predefinito di matrici di rotazione.
    \item \textbf{File di input personalizzato}: un file \texttt{input.txt} precaricato e modificabile dall'utente per inserire manualmente vettori e matrici personalizzati.
\end{itemize}

\paragraph{Dataset dei vettori}
Questo dataset contiene 100 vettori casuali con lunghezza crescente a partire da 5 elementi. I vettori sono generati utilizzando la funzione \texttt{unifrnd} di MATLAB, che genera valori casuali uniformi in questo caso tra 1 e 100. I vettori sono successivamente arrotondati al numero intero più vicino e salvati nel file \texttt{src/data/dataset\_vector.txt}.
Nel caso in cui l'esperimento preveda l'ordinamento di vettori, selezionando questa sorgente da menù, verrà prelevato randomicamnete uno tra i 100 vettori presenti nel dataset.\\
Di seguito il codice MATLAB di generazione.

\begin{verbatim}
    for i=5:104
        v = unifrnd(1,100,1,i);
        v_dataset{i-4} = round(v);
    end
    for i=1:100
        writecell(num2cell(v_dataset{i}),'data_vector.txt','WriteMode','append');
    end
\end{verbatim}

\paragraph{Dataset delle matrici}
Questo dataset contiene 64 possibili matrici di rotazione calcolate sugli angoli prefissati 0, \(\pi/2\), \(\pi\), \(3\pi/2\).
Nel caso in cui l'esperimento preveda la moltiplicazione tra matrici, selezionando questa sorgente da menù, verrà eseguita una traformazione lineare affine combinando una matrice di rotazione e una matrice di scalamento.\\
La matrice di rotazione viene estratta randomicamente tra le 64 matrici presenti nel dataset e la matrice di scalamento viene generata nel \texttt{main.rs} a partire da uno scalare randomico moltiplicato per la matrice identità.
Questo tipo di trasfromazione è presente in numerosi ambiti applicativi come grafica, robotica e computer vision.\\
Di seguito il codice MATLAB di generazione.

\begin{verbatim}
r_1 = [1; 0; 0];
I = [1; 2; 3];
ang = [0 pi/2 pi 3*pi/2];
c = 1;
for i=1:length(ang)
    for j=1:length(ang)
        for k=1:length(ang) 
            T_123 = rot_mat(I,[ang(i),ang(j),ang(k)]);
            T_123_dataset{c} = round(T_123);
            c = c+1;
        end
    end
end

% Check the validity of the rotation matrices created
for i=1:length(T_123_dataset)
    matrix = T_123_dataset{i};
    if isequal(matrix',inv(matrix)) && det(matrix) == 1
        writecell(num2cell(matrix),'data_matrix_rot.txt','WriteMode','append','Delimiter',' ');
        fid = fopen('data_matrix_rot.txt', 'a+');
        fprintf(fid, ' ');
        fclose(fid);
    else
        'matrice non valida'
    end
end
\end{verbatim}

\paragraph{File di input personalizzato}
Il file di input personalizzato \texttt{input.txt} è un file di testo configurabile dall'utente per poter modificare i dati in input. Il file contiene un vettore con la sua relativa lunghezza e una due matrici quadrate con la loro dimensione. \\
I vettori sono rappresentati come sequenze di numeri interi separati da virgola, mentre le matrici sono rappresentate riga per riga con numeri interi separati da spazi.\\
Il file è precaricato con un vettore randomico di 10 elementi e due matrici 4x4. Le due matrici sono rispettivamente una matrice di Wilson e la sua inversa che moltiplicate tra loro danno la matrice identità. La matrice di Wilson è una matrice simmetrica definita positiva e ben condizionata, spesso utilizzata in applicazioni scientifiche e ingegneristiche. \\
Il file permette di essere modificato manualmente dall'utente per inserire i dati desiderati, prestando attentzione ala consistenza tra la dimensione specificata e il numero di elementi presenti.\\

\noindent Di seguito sono descritte le operazioni che precedono la generazione della fault list e l'init dello stage della pipeline preposto a generare le fault entry. Tutto il lavoro viene svolto essenzialmente da un sottomodulo di \texttt{mod fault\_list\_manager} denominato \texttt{mod static\_analysis}.
\subsection{Sottomodulo \texttt{mod static\_analysis}: analisi statica automatica del codice}
Come abbiamo chiarito dall'inizio, il nostro obiettivo -- dopo l'irrobustimento dei dati tramite ridondanza -- è quello di eseguire degli \textit{esperimenti di fault injection} dove i target dei fault sono le variabili coinvolte negli algoritmi scelti come casi di studio. Tuttavia affinché si possa generare una fault list servono delle informazioni che descrivono il codice del caso di studio stesso. Nello specifico serve poter rispondere alle seguenti domande: 
\begin{itemize}
    \itemsep-0.2em
    \item[\ding{52}] \textbf{Quante istruzioni} ha l'algoritmo?
    \item[\ding{52}] \textbf{Quante e quali variabili}?
    \item[\ding{52}] Per ogni variabile, \textbf{qual è la sua etichetta}? La sua \textbf{dimensione}? 
    \item[\ding{52}] Qual è la prima istruzione in cui compare quella variabile? 
\end{itemize}

\noindent
Il sottomodulo \texttt{static\_anlysis} è preposto a rispondere a tutte queste domande. Il codice sorgente viene analizzato dalla \textbf{libreria di parsing} \texttt{syn} (vedi \cite{syn}); questa permette di trasformare una stringa contenente il codice stesso in un \textbf{syntax tree} da cui, utilizzando diversi metodi si possono ricavare le informazioni necessarie. In particolare, al modulo appena citato, vengono affidate in ordine le seguenti operazioni, dato il singolo caso di studio: 
\begin{enumerate}
    \itemsep-0.3em
    \item Lettura da un file di testo della sua implementazione in linguaggio Rust (sono tutte contenute in \newline \texttt{file\_fault\_list/<casodistudio>}), dove 
    \begin{center}
        \texttt{<casodistudio>}$\in$ \texttt{\{selection\_sort, bubble\_sort, matrix\_multiplication\}}
    \end{center}
    \item Trasformazione della stringa contenente il codice in un \textbf{albero di sintassi}; 
    \item Esecuzione di una visita in profondità del  \textbf{syntax tree} ottenuto al fine di ricavare le informazioni richieste; queste poi vengono salvate in una struttura dati di tipo \texttt{ResultAnalysis} di cui riportiamo la definzione dopo;
    \item La struttura dati ottenuta viene infine \textbf{serializzata} in un file \texttt{json} usando il crate \texttt{serde} (si veda \cite{noauthor_serde_nodate} per la documentazione ufficiale) contenuto nella directory corrispondente al caso di studio sotto \texttt{file\_fault\_list}. Così facendo, l'analisi statica del codice può essere fatta indipendentemente dalla generazione della fault list e dall'esperimento di fault injection.
\end{enumerate}

\begin{lstlisting}[language=rust, style=boxed]
#[derive(Serialize, Deserialize, Debug)]
pub struct Variable {
    pub name: String,        //nome 
    pub ty: String,          //tipo       
    pub size: String,        //dimensione
    pub start: usize         //tempo di dichiarazione
}
#[derive(Serialize, Deserialize, Debug)]
pub struct ResultAnalysis{
    pub num_inst: usize,              
    pub vars: Vec<Variable>         
}
\end{lstlisting}

\noindent
Il tipo \texttt{Variable} ha la funzione di descrivere tutte le informazioni legate alla singola variabile. Si fa notare che per restituire una dimensione parametrizzata di una certa variabile (vettori/matrici), il campo \texttt{size} è di tipo \texttt{String}. Per completezza si riporta di seguito un frammento del file risultato dell'analisi statica.

\begin{lstlisting}[language=rust, style=boxed]
"num_inst": 19,
"vars": [{  "name": "a","ty": "Vec < Vec < i32 > >",
            "size": "4*nR*nC","start": 1},
         {  "name": "b","ty": "Vec < Vec < i32 > >",
            "size": "4*nR*nC","start": 1},
        {   "name": "size", "ty": "usize","size": "4",
            "start": 1 }, ...]
\end{lstlisting}

\noindent

\subsubsection{Metodi e descrizione}
Si mostra di seguito la gerarchia delle funzioni presenti in \texttt{mod static\_analysis}, si riporta poi una tabella in cui per ognuna delle funzioni si fornisce una \textit{breve descrizione}.\\
{
\large{
    \dirtree{%
    .1 fn generate\_analysis\_file().
    .2 fn analyze\_function().
    .3 fn count\_statements().
    .4 fn infer\_type\_from\_expr().
    .3 fn extract\_variables().
    .4 fn type\_size().
    }
}
}

\begin{table}[h]
    \centering
    \small
    \begin{tabular}{p{4.5cm} p{11cm}}
        \toprule[1.5px]
        \textbf{Funzione}&\textbf{Descrizione}\\
        \midrule
        \texttt{fn generate\_analysis\_file()}&{
            Funzione wrapper che legge tutto il contenuto del file e lo memorizza in una stringa. Questa dopo essere stata trasformata in un formato compatibile con il parser, diventa l'input della funzione successiva. Per generalizzare quanto più possibile vengono cercate all'interno del codice le funzioni (\texttt{Item::Fn}). Tuttavia all'interno di ogni file c'è il codice di una sola funzione.
        }\\
        \midrule 
        \texttt{fn analyze\_function()}&{
            Chiama le funzioni \texttt{count\_statements()} e \texttt{extract\_variables()} per il conteggio delle istruzioni ed estrazione delle informazioni di variabili associate ai parametri formali della funzione o al corpo della funzione stessa. 
        } \\
        \midrule 
        \texttt{fn count\_statements()}& {
            [Funzione ricorsiva] Conta le istruzioni associati agli statement del \textit{blocco di codice} passato come parametro. Ogni statement può essere di tipo \texttt{Stmt::Local} o \texttt{Stmt::Expr}. Nel primo caso si tratta di un istruzione semplice di cui si possono già estrarre le informazioni sulle variabili usando le funzioni che seguono in questa descrizione. Nel secondo caso si tratta di un blocco di codice composto (\texttt{Expr::If, Expr::While, Expr::ForLoop}) in questo caso viene chiamata \textbf{in modo ricorsivo} la stessa \texttt{fn count\_statements()} e viene passato come input il blocco 'figlio' di questo tipo di espressione.
        } \\
        \midrule 
        \texttt{fn infer\_type\_from\_expr()}&{
            [Funzione ricorsiva] Funzione utilizzata per inferire il tipo di una certa espressione. Questa a sua volta può essere di tipo \texttt{Expr::Lit} (literal) o \texttt{Expr::Assign} (operazione di assegnazione), nel secondo caso viene chiamata ricorsivamente la stessa funzione per inferire il tipo del \textit{Right Hand Side}(RHS). L'output di questa funzione costituisce l'input della funzione \texttt{type\_size()}.
        } \\
        \midrule 
        \texttt{fn extract\_variables()}&{
            Si prende cura di analizzare le informazioni corrispondenti ai \textbf{parametri della funzione}.
        } \\
        \midrule 
        \texttt{fn type\_size()}&{
            Effettua il binding tipo-dimensione. Utilizzata da \texttt{count\_statement()} per popolare il campo \texttt{size} della struttura di tipo \texttt{ResultAnalysis}.
        } \\
        \bottomrule[1.5px]
    \end{tabular}
    \caption{\texttt{mod static\_analysis()}}
\end{table}

\subsubsection{Un piccolo CAVEAT per la scrittura del codice}
Poiché il modulo di analisi statica prende in pasto un file con il codice dell'algoritmo (non irrobustito), questo deve essere scritto prestando attenzione ad un piccolo particolare che se non rispettato potrebbe far fallire l'inferenza del tipo. Più nello specifico, nella dichiarazione di una variabile, se questa viene ricavata tramite assegnazione di un'altra variabile già esistente, bisogna utilizzare un'\textbf{annotazione esplicita di tipo}. Si mostra di seguito un esempio: 
\begin{lstlisting}[language=rust, style=boxed]
let mut a=13; 
let mut b=15; 
let mut c:i32 =a+c; 
\end{lstlisting}
Mentre nei primi due casi il tipo viene inferito direttamente\footnote{
    Nella funzione \texttt{infer\_type\_from\_expr()} mi accorgo che ho un \texttt{Expr::Assign}, quindi chiamo ricorsivamente la stessa funzione che a questo punto trova la \texttt{Expr::Literal} associata all'intero e viene ritornato il tipo corrispondente.
}, nella terza istruzione bisogna aggiungerlo esplicitamente perché questo dipende da un'altra variabile. Questo dettaglio non è stato implementato per non aggiungere complessità ad un codice già non banale.


\section{Fault List Manager}\label{sec:FLM}
Dopo un percorso abbastanza articolato, abbiamo sviluppato \textit{un modo automatico} per ottenere le informazioni sulle variabili facendo passare il codice sorgente del caso di studio attraverso i metodi dell'analisi statica. \\
Il \textit{glossario} che viene dall'output di questa fase è di cruciale importanza per la generazione della fault list descritta in questo paragrafo. \\
Prima di entrare in merito della discussione è doveroso aggiungere un dettaglio non trascurabile. Gli algoritmi (ordinamento/moltiplicazione di matrici) possono lavorare su \textbf{dati diversi} ad ogni esecuzione. Per cui per sapere effettivamente il numero di istruzioni che quel set di dati genera bisognerebbe eseguire quel codice! A questo scopo sono state implementate delle funzioni denominate \texttt{run\_for\_count\_<casodistudio>()} a cui è associata  un'esecuzione fittizia del codice in analisi. Il loro output è un intero associato al numero di istruzioni che un certo algoritmo applicato ad un certo set di dati (matrice o vettore) produce.\\

\noindent
Abbiamo accennato nell'introduzione che il modello di guasto che vogliamo adottare è quello del \textbf{single bit-flip}. Durante la vita dell'applicativo si possono verificare un certo numero di fault. Per simularne l'occorrenza ed eseguire gli esperimenti sul codice modificato, viene generata randomicamente una lista di guasti.

\subsection{Fault list entry}
La struttura dati preposta a contenere le informazioni sul singolo fault è la seguente: 
\begin{lstlisting}[language=rust, style=boxed]
#[derive(Debug, Serialize, Deserialize, Clone)]
pub struct FaultListEntry{
    pub var: String,
    pub time: usize,
    pub flipped_bit: usize,
}
\end{lstlisting}
dove \texttt{var} è il \textbf{nome della variabile}, \texttt{time} è il \textbf{tempo di iniezione}, mentre \texttt{flipped\_bit} è il bit della variabile che è stato modificato dal guasto. Sul tipo \texttt{FaultListEntry} sono derivati, tra gli altri, i tratti \texttt{Serialize} e \texttt{Deserialize} utili per il \textit{marshalling
/unmarshalling}\footnote{
    Sono le operazioni tramite le quali la struttura viene portata sul file (\textit{marshalling}) e dal file viene riportata in memoria (\textit{unmarshalling}).
} della fault list su file.

\subsection{Generazione della fault list}
L'algoritmo di generazione della  fault list è riportato di seguito.

\begin{algorithm}
    \caption{Algoritmo di \textit{Generazione della fault list}}
    \begin{algorithmic}
        \For {\texttt{i} in [0, NUM\_FAULT]}
            \State $\underline{var} \gets$ rand(1,NUM\_VAR)\Comment{Scelgo una variabile tra quelle disponibili}
            \If {var.type == 'matrix'}
                \State $r \gets$ rand(0,var.nR) \Comment{Seleziono la riga}
                \State $c \gets$ rand(0,var.nC) \Comment{Seleziono la colonna}
                \State $\underline{var} \gets$ mat[r][c]
                \State $\underline{flipped\_bit} \gets$ rand(0, var.size-1) \Comment{Scelgo il bit da flippare}
                \State $\underline{time} \gets$ rand(var.start, N\_INST)\Comment{Scelgo il tempo di iniezione}
            \ElsIf {var.type == 'vector'}
                \State $index \gets$ rand(0, var.len)
                \State $var \gets$ vec[index]
                \State $\underline{flipped\_bit} \gets$ rand(0, var.size-1) \Comment{Scelgo il bit da flippare}
                \State $\underline{time} \gets$  rand(var.start, N\_INST) \Comment{Scelgo il tempo di iniezione}
            \Else
                \State $\underline{flipped\_bit} \gets$ rand(0, var.size-1) \Comment{Scelgo il bit da flippare}
                \State $\underline{time} \gets$  rand(var.start, N\_INST) \Comment{Scelgo il tempo di iniezione}
            \EndIf \\
            FaultList.insert($var.name,\ time, \  flipped\_bit$) \Comment{Creo la fault entry e la inserisco nella fault list}
        \EndFor
    \end{algorithmic}
\end{algorithm}
Alla fine della sua creazione, la fault list viene serializzata in formato \texttt{json} alla stregua di quanto fatto per il glossario di analisi statica. La costante NUM\_FAULT se non settata dall'utente da linea di comando viene impostata a 2000, mentre la costante N\_INST è l'output delle funzioni \texttt{run\_for\_count()}. Il tempo di iniezione viene scelto in modo compatibile, nel senso che non posso iniettare in una variabile se prima questa non è stata dichiarata, ecco perché si è scelto di ricavare durante l'analisi dell'albero di sintassi anche l'informazione memorizzata in \texttt{Variable::start}. Riportiamo uno stralcio di fault list nello snipppet che segue:

\begin{lstlisting}[language=rust, style=boxed]
    {   "var": "swapped",   "time": 218,    "flipped_bit": 6    },
    {   "var": "j",         "time": 15,     "flipped_bit": 2    },
    {   "var": "swapped",   "time": 102,    "flipped_bit": 4    },
    {   "var": "n",         "time": 20,     "flipped_bit": 24   },
\end{lstlisting}

\subsection{Stage pipeline}
Una volta creata la fault list, possiamo far partire l'\textbf{esperimento di fault injection}. Lo stage della pipeline associata al \textsf{Fault list manager} è espletato dalla seguente funzione: 

\begin{lstlisting}[language=rust, style=boxed]
pub fn fault_manager(tx_chan_fm_inj: Sender<FaultListEntry>, fault_list:String){
    let flist_string = fs::read_to_string(fault_list).unwrap();
    let flist:Vec<FaultListEntry>=serde_json::from_str(&flist_string.trim()).unwrap();
    flist.into_iter().for_each(|el|tx_chan_fm_inj.send(el).unwrap());
    drop(tx_chan_fm_inj);
}
\end{lstlisting}
Gli input di questo stage sono il path del file in cui è contenuta la fault list (data source) e l'estremità del canale (pipe) verso l'iniettore. Sono eseguite in ordine le seguenti operazioni: 
\begin{enumerate}
    \itemsep-0.3em
    \item Lettura del file di testo in una stringa (\texttt{flist\_string});  
    \item Deserializzazione della stringa in una collezione \texttt{Vec<FaultListEntry>} usando la funzione\newline \texttt{serde\_json::from\_str()}
    \item La collezione viene trasformata in un \textbf{iteratore} di \texttt{FaultListEntry}. Ogni elemento estratto da questo iteratore tramite il metodo \texttt{for\_each()} viene mandato nel canale verso l'iniettore che lo utilizzerà in modo opportuno.
\end{enumerate} 
\section{Injector}\label{sec: Injector}
\subsection{Aspetti Generali}
L'iniettore è stato pensato come un componente della pipeline che riceve le fault list entry dal fault list manager, utilizzandole poi per iniettare gli errori nel momento 
corretto durante l'esecuzione dell'algoritmo tesato. Il risultato dell'esecuzione viene poi utilizzato per creare il TestResult relativo alla singola fault list entry e passato 
al successivo stadio della pipeline. 

Per l'implementazione dell'iniettore vengono utilizzati 2 thread, uno per l'esecuzione dell'algoritmo che chiameremo \textit{runner}, e uno per l'esecuzione dell'i\-niettore che 
chiameremo \textit{injector}. I due thread condividono le variabili in uso che, durante un'istanza dell'esecuzione dell'algoritmo sotto esame (un'istanza per ciascuna fault list 
entry), verranno lette e modificate da entrambi i thread: il thread runner leggerà e modificherà le variabili seguendo l'ordine delle istruzioni dell'algoritmo, il thread
injector leggerà la variabile su cui iniettare l'errore per poter calcolare il nuovo valore (ovvero quello contenente l'errore) e modificandola di conseguenza. Affinché i due 
thread si sincronizzino correttamente e l'iniezione dell'errore avvenga nell'istante specificato nella fault list entry, i due thread utilizzano 2 canali monodirezionali \textit{mpsc} in modo che 
dopo ogni istruzione dell'algoritmo eseguita dal runner venga mandato un messaggio all'injector su un canale e ne venga attesa la risposta sull'altro.

\subsection{Aspetti tecnici}
\subsubsection{Injector Manager}
La funzione chiamata \textit{injector\_manager} ha la funzione di coordinare la ricezione delle fault list entry provenienti dallo stato precedente della pipeline tramite un canale dedicato, ricevendo anche il canale per trasmettere i risultati, l'algoritmo target e i dati da usare durante l'analisi.

\begin{lstlisting}[language=Rust, style=boxed]
pub fn injector_manager(rx_chan_fm_inj: Receiver<FaultListEntry>,
                tx_chan_inj_anl: Sender<TestResult>,
                target: String,
                data: Data<i32>);
\end{lstlisting}

Al suo interno la funzione tramite un ciclo while attende la ricezione sul canale delle fault list entry e, per ciascuna, crea il set di variabili utilizzate (in base al tipo di algoritmo in esecuzione), i 2 canali con cui i thread gestiranno la sincronizzazione e i 2 thread \textit{runner} e \textit{injector}.

Affinche' siano testabili più algoritmi, ciascuno avente il proprio set di variabili che utilizza, è stata usata un'enum chiamata \textit{AlgorithmVariables} contenente per ciascun algoritmo una struct contenente le variabili.

\begin{lstlisting}[language=Rust, style=boxed]
enum AlgorithmVariables {
    SelectionSort(SelectionSortVariables),
    BubbleSort(BubbleSortVariables),
    MatrixMultiplication(MatrixMultiplicationVariables),
}
\end{lstlisting}

Le struct relative ai singoli algoritmi contengono, per ogni variabile, un \textit{RwLock} contenente a sua volta il tipo \textit{Hardened} corrispondente. Dovendo condividere questa struttura tra più thread eseguiti, era necessario renderla accessibile in modo sicuro (dovendo essere sia letta che scritta) e per questo motivo una possibile soluzione era quella di racchiudere la struttura per intero all'interno di un \textit{Mutex} o \textit{RwLock}. Questa soluzione presentava però delle criticità. Per effettuare il controllo condizionale per i cicli while era richiesto di acquisire il lock prima del check sulla condizione del ciclo, ma una volta acquisito il lock fuori dal ciclo questo veniva mantenuto per l'intera durata del ciclo, impedendo all'\textit{injector} di iniettare l'errore su una delle variabili. Di conseguenza l'opzione migliore e che richiedesse meno overhead a livello di codice era racchiudere ciascuna singola variabile della struct in un RwLock anziché la struttura per intero. La scelta di utilizzare RwLock è stata motivata principalmente da una possibile migliore gestione delle read e write, dovuta a numero di letture e scrittura sbilanciato in base all'algoritmo eseguito.

\begin{lstlisting}[language=Rust, style=boxed]
struct SelectionSortVariables {
    i: RwLock<Hardened<usize>>,
    j: RwLock<Hardened<usize>>,
    N: RwLock<Hardened<usize>>,
    min: RwLock<Hardened<usize>>,
    vec: RwLock<Vec<Hardened<i32>>>,
}
\end{lstlisting}

Una volta creata la struct contenente le variabili della fault list entry corrente, vengono aperti i canali di comunicazione tra \textit{runner} e \textit{injector} ed eseguiti i rispettivi due thread. Quando il thread \textit{runner} termina invia all'analizzatore (stadio di pipeline successivo) i risultati ottenuti.

\subsubsection{Runner}
Il thread \textit{runner} esegue una funzione wrapper chiamata \textit{runner} la quale si occupa di lanciare l'esecuzione dell'algoritmo irrobustito corretto per il tipo di analisi che si sta facendo e gestendo il risultato prodotto da questo. 

\begin{lstlisting}[language=Rust, style=boxed]
fn runner(variables: Arc<AlgorithmVariables>,
          fault_list_entry: FaultListEntry,
          tx_runner: Sender<&str>,
          rx_runner: Receiver<&str>) -> TestResult
\end{lstlisting}

In base al tipo di algoritmo \textit{target} sono stati creati degli algoritmi ad-hoc per poter interagire correttamente con l'iniettore. Questi sono delle versioni rivisitate delle versioni irrobustite originali, le quali non sarebbero state in grado di sincronizzarsi con l'iniettore per subire i fault. Di seguito viene descritta la struttura di questi algoritmi, facendo esempi relativi al Selection Sort, in quanto gli altri seguono tutti la stessa logica. 

\paragraph{Algoritmo Testato}
Ciascun algoritmo riceve le variabili da utilizzare, il canale su cui trasmettere il completamento di un'istruzione e quello su cui attendere l'eventuale inserimento del fault.

\begin{lstlisting}[language=Rust, style=boxed]
pub fn runner_matrix_multiplication(variables: &MatrixMultiplicationVariables, 
            tx_runner: Sender<&str>, 
            rx_runner: Receiver<&str>) -> Result<(), IncoherenceError>
\end{lstlisting}

La procedura per l'esecuzione di una qualsiasi istruzione è:
\begin{itemize}
    \item Accesso al lock con conseguente lettura/scrittura della variabile
    \item Scrittura sul canale \textit{tx\_runner} per comunicare all'\textit{injector} che un'istruzione è stata eseguita
    \item Attesa sul canale \textit{rx\_runner} che l'\textit{injector} termini le sue operazioni, necessario affinché \textit{runner} e \textit{injector} rimangano sincronizzati
\end{itemize}

Riportiamo di seguito un esempio di un'istruzione equivalente all'istruzione $j.assign((i+1)?)?$:
\begin{lstlisting}[language=Rust, style=boxed]
// j = i + 1   -- versione non irrobustita
// j.assign((i+1)?)?    -- versione irrobustita
variables.j.write().unwrap().assign((*variables.i.read().unwrap() + 1)?)?;
tx_runner.send("").unwrap();
rx_runner.recv().unwrap();
\end{lstlisting}

L'algoritmo ritorna un Result, contenente:
\begin{itemize}
    \item \textit{Ok(...)}: successo ed esecuzione portata a termine correttamente; questo contiene il risultato dell'algoritmo (ad esempio il vettore ordinato o il risultato della moltiplicazione delle matrici);
    \item \textit{Err$<$IncoherenceError$>$}: variante dell'enum \textit{IncoherenceError} che descrive il tipo di errore riscontrato 
\end{itemize}

\paragraph{Terminazione Runner}
Il runner termina eseguendo un pattern match sul risultato dell'algoritmo eseguito, producendo il TestResult che verrà utilizzato dall'analizzatore per ottenere statistiche utili.

\subsection{Injector}
L'\textit{injector} è una funzione che si occupa di iniettare nel momento corretto il fault contenuto nella fault list entry sulla variabile indicata. Per fare ciò, riceve le variabili usate dall'algoritmo e condivise con il \textit{runner}, la fault list entry e i canali necessari per la sincronizzazione con il \textit{runner}.

\begin{lstlisting}[language=Rust, style=boxed]
fn injector(variables: Arc<AlgorithmVariables>, 
            fault_list_entry: FaultListEntry,
            tx_injector: Sender<&str>,
            rx_runner: Receiver<&str>)
\end{lstlisting}

L'informazione sul tipo di algoritmo in esecuzione è ricavata dal tipo di variabili ricevute, essendo queste un'istanza dell'enum \textit{AlgorithmVariables}. Viene poi manutenuto un \textit{counter} necessario a contare il numero di istruzioni eseguite per poi al momento indicato nella fault list entry iniettare l'errore. Tramite un ciclo while, che termina quando il canale condiviso con il \textit{runner} viene chiuso, vengono ricevuti gli impulsi che indicano la terminazione di un'istruzione. Il flusso di operazioni eseguite è: 

\begin{enumerate}
    \item Calcola la maschera in base al bit indicato nella fault list entry
    \item Per ogni segnale ricevuto dal \textit{runner}:
    \begin{enumerate}
        \item Incrementa il counter
        \item Se $\textit{counter} == \textit{fault\_list\_entry}.\textit{time}$
        \begin{enumerate}
            \item Ricava la variabile su cui iniettare contenuta nella fault list entry
            \item Tramite match inietta sulla variabile la maschera calcolata
        \end{enumerate}
        \item Manda sul canale verso il \textit{runner} il segnale per la continuazione della sua esecuzione
    \end{enumerate}
\end{enumerate}

Le maschere vengono calcolate come $\textit{mask} = 2^{\textit{fault\_mask}}$. Le maschere vengono applicate alle variabili tramite XOR. Prendendo un esempio per quanto riguarda il Selection Sort:
\begin{lstlisting}[language=Rust, style=boxed]
"i" => {
    let val = var.i.read().unwrap().inner().unwrap().clone();   // leggo il valore della variabile
    let new_val = val \wedge mask;                                   // nuovo valore da salvare (XOR per il bitflip)
    var.i.write().unwrap()["cp1"] = new_val;                    // inietto l'errore
}
\end{lstlisting}
















        
\section{Analizzatore}\label{sec:analizer}
L'analizzatore si colloca come elemento conclusivo della pipeline di \textit{fault injection}. La sua funzione principale è raccogliere e organizzare i risultati generati durante l'iniezione di fault negli algoritmi sottoposti a test al fine di fornire poi una visione dettagliata del comportamento degli algoritmi irrobustiti e non, in presenza di fault.\\
Tali dettagli sono riassunti in un report in formato pdf generato dinamicamente in base ai risultati ottenuti. 

\subsection{Struct Analyzer e Faults}
Per memorizzare e gestire i dati rilevanti, è stata progettata una struttura dati denominata \textbf{Analyzer} che viene presentata di seguito:
\begin{lstlisting}[language=rust, style=boxed]
    #[derive(Serialize,Deserialize,Debug,Clone)]
    pub struct Analyzer{
        pub(crate) n_esecuzione: i8,
        pub(crate) faults: Faults,
        pub(crate) input: Data<i32>,
        pub(crate) output: Data<i32>,
        pub(crate) time_experiment: f64,
        pub(crate) time_alg_hardened: f64,
        pub(crate) time_alg_not_hardened: f64,
        pub(crate) byte_hardened: f64,
        pub(crate) byte_not_hardened: f64,
        pub(crate) target_program: String,
    }
\end{lstlisting}

Al suo interno, il campo \textit{faults} è di tipo \textbf{Faults}, una struttura che include una serie di contatori dedicati. Questi contatori vengono incrementati ogni volta che un fault specifico viene rilevato sul canale di comunicazione tra l'iniettore e l'analizzatore. Di seguito sono riportati i contatori presenti nella struttura \textbf{Faults}:
\begin{lstlisting}[language=rust, style=boxed]
    #[derive(Serialize,Deserialize,Debug,Clone)]
    pub struct Faults{
        pub(crate) n_silent_fault: usize,
        pub(crate) n_assign_fault: usize,
        pub(crate) n_inner_fault: usize,
        pub(crate) n_sub_fault: usize,
        pub(crate) n_mul_fault: usize,
        pub(crate) n_add_fault: usize,
        pub(crate) n_indexmut_fault: usize,
        pub(crate) n_index_fault: usize,
        pub(crate) n_ord_fault: usize,
        pub(crate) n_partialord_fault: usize,
        pub(crate) n_partialeq_fault: usize,
        pub(crate) n_fatal_fault: usize,
        pub(crate) total_fault: usize,
    }
\end{lstlisting}

Un'attenzione particolare è dedicata ai fault silent i quali rappresentano errori iniettati durante l'esecuzione di un algoritmo ma non intercettati dal sistema irrobustito. Sebbene la maggior parte di questi fault non abbia un impatto diretto sull'output del sistema, una piccola percentuale (circa il 10\%) può generare risultati errati, evidenziando casi critici in cui il sistema irrobustito fallisce nel mantenere l'integrità dell'elaborazione.

\subsection{Funzionalità dell'analizzatore}
Per semplicità possiamo affermare che per ogni fault iniettato, in generale l'analizzatore distingue le due seguenti macrocategorie:
\begin{itemize}
    \item \textbf{Fault silent}: rappresentano gli errori non intercettati dal sistema irrobustito.
    \item \textbf{Fault identificati}: corrispondono agli errori rilevati dal sistema irrobustito, categorizzati in base all'operazione specifica che li ha generati (ad esempio, operazioni di assegnazione, somma o moltiplicazione).
\end{itemize}

In aggiunta alla rilevazione e categorizzazione degli errori, l'analizzatore tiene traccia di metriche chiave legate all'overhead introdotto dall'irrobustimento del codice, come:
\begin{itemize}

    \item \textbf{Tempi di esecuzione}: confronto tra i tempi necessari per eseguire il codice irrobustito e quello non irrobustito.
    \item \textbf{Dimensione del file}: differenza nella dimensione dei file contenenti il codice irrobustito e non irrobustito.
    
\end{itemize}

\subsection{Tipologie di analisi}
L'analizzatore supporta tre modalità principali di analisi, ognuna delle quali genera un report in formato PDF, salvato nella cartella results. Di seguito una panoramica:
\begin{enumerate}
\item Analisi singola:
    \begin{itemize}
        \item Analizza un singolo algoritmo su cui vengono iniettati un     numero prefissato di fault.
        \item Produce un file PDF denominato \textless \textit{nome\_file}\textgreater.pdf.
    \end{itemize}

\item Analisi su più algoritmi:
    \begin{itemize}
        \item Valuta il comportamento di tre algoritmi diversi:     selection sort, bubble sort e matrix multiplication.
        \item Produce un file PDF denominato \textless \textit{nome\_file}\textgreater\_all.pdf.
    \end{itemize}

\item Analisi su diverse cardinalità:
    \begin{itemize}
        \item Analizza un singolo algoritmo utilizzando tre diverse cardinalità della lista di fault (1000, 2000 e 3000 fault).
        \item Produce un file PDF denominato \textless \textit{nome\_file}\textgreater\_diffcard.pdf.
    \end{itemize}
\end{enumerate}
Ogni report include informazioni dettagliate sui fault rilevati e non rilevati, insieme alle metriche di performance e dimensioni del codice. Questo sistema di analisi offre una visione completa dell'efficacia del processo di irrobustimento e del relativo impatto su risorse e prestazioni.    
\section{PDF Generator}\label{sec:pdfgenerator}
Il modulo \textit{pdf\_generator} si occupa di generare dinamicamente un report in formato PDF contenente i risultati dell'analisi dei fault iniettati. Per facilitarne la lettura e la comprensione, ogni report è composto da una serie di tabelle e grafici che riassumono i risultati ottenuti durante l'esecuzione dell'algoritmo irrobustito e non.\\
Per ogni tipologia di analisi precedentemente esposta viene generato un report specifico.

Per costruire il report, è stata utilizzata la libreria \textit{genpdf} che non è altro che un wrapper della libreria \textit{printpdf} che permette, utilizzando funzioni di alto livello,di creare documenti PDF personalizzati a partire direttamente da codice Rust.\\

Un altro motivo per cui è stata scelta questa libreria è la possibilità di inserire immagini in formato PNG in maniera semplice. Nel caso specifico le immagini sono anch'esse generate automaticamente durante l'esecuzione del programma e rappresentano grafici a torta o a barre che riassumono i dati raccolti durante l'analisi.\\ 
Queste immagini vengono come prima cosa generate utilizzando la libreria \textit{charts-rs} che data una stringa contenete il json del grafico da creare, restituisce una stringa contenete una descrizione del grafico in formato SVG. Per questo motivo, nella seconda fase, avviene la chiamata alla funzione \textit{svg\_to\_png()} contenuta all'interno del file \textit{encoder.rs}. Questa funzione converte una stringa con il codice SVG in un'immagine PNG salvandola poi in un'apposita cartella.\\

\subsection{Costruzione del report}
Come già detto il report è personalizzato in base alla tipologia di analisi condotta, ma in generale le funzioni utilizzate per la creazione e gestione del report sono le stesse.\\
Di seguito vengono elencate le principali funzioni utilizzate per la creazione del report:
\begin{enumerate}[label=\Alph*.]
    \item \textbf{Configurazione del Documento PDF}
    \begin{itemize}
        \item \textit{setup\_document}: Configura l'aspetto generale del PDF, inclusi margini, stile del testo e numero di pagina. Imposta il titolo principale "Report" in stile evidenziato.
    \end{itemize}
    \item \textbf{Descrizione Testuale dei Dati}
    \begin{itemize}
        \item \textit{get\_list\_input\_output}: Genera una descrizione testuale degli input e degli output degli algoritmi analizzati. Supporta vettori e matrici, creando elenchi puntati per migliorare la leggibilità del documento.
    \end{itemize}
    \item \textbf{Creazione di Tabelle}
    \begin{itemize}
        \item \textit{gen\_table\_faults}: Genera una tabella che mostra i dati relativi ai guasti per ciascun algoritmo. Struttura le righe in base ai nomi degli algoritmi (side\_headers) e le colonne con intestazioni (top\_headers).
        \item \textit{gen\_table\_dim\_time}: Crea una tabella che visualizza le dimensioni dei dati (ad esempio, byte elaborati con e senza protezione) e i tempi di esecuzione (con e senza protezione). Ogni riga rappresenta un algoritmo (determinato da side\_headers).
    \end{itemize} 
    \item \textbf{Generazione dei Grafici}
    \begin{itemize}
        \item \textit{gen\_pie\_chart}: Crea grafici a torta per visualizzare la distribuzione dei guasti (faults) di un algoritmo. Restituisce i percorsi delle immagini PNG generate.
        \item \textit{gen\_bar\_chart}: Crea un grafico a barre che mostra la percentuale di guasti rilevati per ogni algoritmo analizzato e lo salva come "\textit{percentage\_detected.png}".
    \end{itemize}
    \item \textbf{Inserimento di Immagini nel PDF}
    \begin{itemize}
        \item \textit{add\_image\_to\_pdf}: Inserisce immagini nel report PDF. Supporta l'inserimento di singole immagini (analisi singola) o più immagini disposte in una tabella (analisi su più algoritmi o diverse cardinalità).
    \end{itemize}   
\end{enumerate}


    
    
    
     


\newpage
\printbibliography[title=Riferimenti bibliografifi]


\end{document}
