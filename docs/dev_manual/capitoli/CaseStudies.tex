\section{Casi di studio}\label{sec:CaseStudies}
I tre casi di studio presi in considerazione sono: 
\begin{itemize}
    \item Selection Sort
    \item Bubble Sort
    \item Moltiplicazioni tra matrici
\end{itemize}
Per ciascuno di essi vengono implementate due versioni:
\begin{itemize}
    \item La versione Non Hardened, implementazione standard del caso studio.
    \item La versione Hardened, basata sul tipo Hardeded e sulle tre regole di trasformazione del codice precedentemente enunciate.
\end{itemize}
Essendo la ridondanza dei dati e le operazioni associate integrate nell'implementazione del tipo Hardened, per tutti i casi studio le due versioni non differiscono nella strategia algoritmica, ma solamente in piccole varaiazioni sintattiche.
\\ La logica di ordinamento e di calcolo infatti rimane sempre la stessa. \\ \\
Di seguito vengono riportati i codici dei casi studio.
\subsection{Selection Sort}
L'algoritmo di ordinamento selection sort ordina un vettore trovando il minimo in ogni iterazione e scambiandolo con l'elemento corrente.
\\ \\Versione Non Hardened:
\begin{lstlisting}[language=rust, style=boxed] 
    pub fn selection_sort(mut vet:Vec<i32>)->Vec<i32>{

    let n:usize = vet.len();
    let mut j=0;
    let mut min=0;

    let mut i=0;
    while i< n -1{
        min=i;
        j=i+1;

        while j< n {
            if vet[j] < vet[min]{ min=j; }
            j = j+1;
        }
        vet.swap(min,i);
        i=i+1;
    }
    vet
}
\end{lstlisting}
Versione Hardened:
\begin{lstlisting}[language=rust, style=boxed] 
pub fn selection_sort(vet: &mut Vec<Hardened<i32>>)->Result<(), IncoherenceError>{

    let n:Hardened<usize> = vet.len().into();
    let mut j= Hardened::from(0);
    let mut min = Hardened::from(0);

    let mut i= Hardened::from(0);
    while i<(n -1)?{
        min.assign(i)?;                 
        j.assign((i+1)?)?;        
      
        while j< n {
            if vet[j]<vet[min]  {   min.assign(j)?; }
            j.assign((j+1)?)?;
        }
        vet.swap(i.inner()?, min.inner()?);
        i.assign((i+1)?)?;
    }
    Ok(())
}
\end{lstlisting}

\subsection{Bubble Sort}
L'algoritmo di ordinamento bubble sort è un algortimo che confronta e scambia elementi adiacenti finché i dati non risultano ordinati. La variabile swapped ottimizza il processo in quanto se non vengono effettuati scambi in un ciclo, significa che il vettore è già ordinato.\\
\\Versione Non Hardened:
\begin{lstlisting}[language=rust, style=boxed] 
    pub fn bubble_sort(mut vet: Vec<i32>) -> Vec<i32> {
        let n:usize = vet.len();
        let mut i = 0;
    
        while i < n {
            let mut swapped = false;
            let mut j = 0;
    
            while j < n - i - 1 {
                if vet[j] > vet[j + 1] {
                    vet.swap(j, j + 1);
                    swapped = true;
                }
                j += 1;
            }
            if !swapped {
                break;
            }
            i += 1;
        }
        vet
    }    
\end{lstlisting}
Versione Hardened:
\begin{lstlisting}[language=rust, style=boxed] 
pub fn bubble_sort(vet: &mut Vec<Hardened<i32>>) -> Result<(), IncoherenceError> {

    let n = Hardened::from(vet.len());
    let mut i = Hardened::from(0);

    while i < n {
        let mut swapped = Hardened::from(false);
        let mut j = Hardened::from(0);

        while j < ((n - i)? - 1)? {
            if vet[j].inner()? > vet[(j + 1)?].inner()? {
                vet.swap(j.inner()?, (j + 1)?.inner()?);
                swapped = Hardened::from(true);
            }
            j.assign((j + 1)?)?;
        }
        if !swapped.inner()? {
            break;
        }
        i.assign((i + 1)?)?;
    }
    Ok(())
}
\end{lstlisting}



\subsection{Moltiplicazioni tra matrici}
L'algoritmo moltiplica due matrici quadrate, calcolando ogni elemento come il prodotto scalare delle righe di una matrice e delle colonne dell'altra.
\\ \\Versione Non Hardened:
\begin{lstlisting}[language=rust, style=boxed] 
    pub fn matrix_multiplication(a: Vec<Vec<i32>>, b: Vec<Vec<i32>>) -> Vec<Vec<i32>> {
        let size:usize = a.len();
        let mut result: Vec<Vec<i32>> = Vec::new();
    
        let mut i = 0;
        let mut j = 0;
        let mut k = 0;
    
        while i < size {
            let mut row: Vec<i32> = Vec::new();
            j = 0;
    
            while j < size {
                let mut acc = 0;
                k = 0;
    
                while k < size {
                    acc += a[i][k] * b[k][j];
                    k += 1;
                }
                row.push(acc); 
                j += 1;
            }
            result.push(row);
            i += 1;
        }
        result
    }    
\end{lstlisting}
Versione Hardened:
\begin{lstlisting}[language=rust, style=boxed] 
    pub fn matrix_multiplication(a: &Vec<Vec<Hardened<i32>>>, b: &Vec<Vec<Hardened<i32>>>) -> Result<Vec<Vec<Hardened<i32>>>, IncoherenceError> {
    
        let size = Hardened::from(a.len());
        let mut result: Vec<Vec<Hardened<i32>>> = Hardened::from_mat(Vec::new());
    
        let mut i = Hardened::from(0);
        let mut j = Hardened::from(0);
        let mut k = Hardened::from(0);
    
        while i < size {
            let mut row: Vec<Hardened<i32>> = Vec::new();
            j.assign(Hardened::from(0))?;
    
            while j < size {
                let mut acc = Hardened::from(0);
                k.assign(Hardened::from(0))?;
    
                while k < size {
                    acc.assign((acc + Hardened::from(a[i.inner()?][k.inner()?].inner()?   *   b[k.inner()?][j.inner()?].inner()?) )? )?;
                    k.assign((k + 1)?)?;
                }
                row.push(acc); 
                j.assign((j + 1)?)?;
            }
            result.push(row); 
            i.assign((i + 1)?)?;
        }
        Ok(result)
    }
\end{lstlisting}